\section{Sammanfattning och Diskussion}

Systemet som utvecklades motsvarar de utsatta målen\todo{byt ut målet}, systemet tillåter filtrering av innehåll på basis av information som samlas in vid ett besök, vilket var det mest grundläggande målet för systemet. Den tekniska arkitekturen är lätt och flexibel, den tillåter systemet att enkelt skala upp och ner i enlighet med behov. Mjukvaruarkitekturen möter behoven på mjukvaran i det skede som den är och systemet är öppet för vidareutveckling.

\subsection{Idéer för vidare utveckling}

SmartElement har genom projektet utvecklats till ett MVP (Minimum-Viable-Produkt) stadie. Det stöder den mest grundläggande funktionaliteten som krävs. Det finns dock en del områden var systemet skulle kunna utvecklas och tekniker som skulle kunna användas för förbättring av systemet.

\subsubsection{Poängsättning av material}

Ett system för poängsättning av material som besökaren ser har planerats, men inte implementerats. Genom att använda anpassad information och de anpassade filtret kan man dock i dagens läge skapa en rudimentär implementation av ett poängsättningssystem. Med dedikerat stöd för poängsättning skulle användare kunna definiera kundsegment baserat på olika mätare och systemet skulle kunna använda statistikinformationen för att klassificera besökare som del av dessa.

\subsubsection{Förutsägning}

I dagens läge är SmartElements filtreringssystem reaktivt. Det reagerar på historisk information. Genom att koppla in ett system för förutsägning skulle man kunna förutse relevans. Systemet samlar redan in en del data som skulle kunna användas för förutsägning av relevans och genom att träna ett system med information specifik för en webbsida, kunde man potentiellt förutsäga vilka innehåll en ny besökare potentiellt är intresserad av.

\todo[inline]{Wrap it up}

% vim: set tw=78:ts=2:sw=2:et:fdm=marker:wrap:wm=78:ft=tex
% vim: spell spelllang=sv
