\section{Sammanfattning och Diskussion}

Sammanfattning av arbetet, tillbakatitt på produkten. Diskussion om förbättringar som skulle kunna göras i systemet. Diskussion om tekniker som kunde bytas ut för att eventuellt förbättra prestanda. Deiskussion om svårigheter med produkten.

\section{För uppgiften Utvidgad disposition}

Istället för att skapa ett skillt dokument valde jag att skriva använda mig av dokumentet som kommer att bli mitt slutarbete. Här kan man se en grund för arbetet och ett preliminärt skelett för dokumentet.

Jag har påbörjat arbetet med att skriva en kort bakgrund om projektet samt en kort introduktion av systemet som hoppeligen kommer att hjälpa läsaren förstå olika val som kommer att förklaras ki arkitektur delen. Jag har även förökt skriva ned några korta meningar under alla rubriker för att ge en bild av det tänkta innehållet.

Källförteckningen är kanske lite kort men den innehåller två relativt tunga verk, den så kallade Gang of Four boken (Design Patterns: Elements of Reusable Object-Orientated Software) och Introduction to algorithms, vilka i samband med The Pragmatic Programmer och High Performance Mysql ganska långt täcker arkitekturen av projektet. Referenserna sträcker sig inte så långt ennu pga de skrivna kapitlens natur, bakgrunden har någon referens till tidsskrifter som behandlat personalisering. Jag skulle dock säga att process och arkitektur kapitlen kommer vara de som behöver mest referenser, eftersom det är där jag gjort ställningstaganden och val som kan behöva stödas. Stilen är Harvard.

Av källorna jag valt är Design Patterns och Introduction to Algorithms båda rätt kompletta verk inom respektive fält så de kan anses vara rätt trovärdiga. The pragmatic programmer är en bok som titt som tätt rekommenderas inom mjukvarubranschen, den innehåller många beprövade metoder för mjukvaruutvekling som är i bruk världen över. Utöver att söka böcker så har jag användt mig av Association for Computing Machinerys söktjänst för att söka information samt försökt använda nelliportalen, ACM kan anses vara ganska trovärdig källa som en av världens äldsta föreningar för datavetenskap. Den information som tagits från webben är närmast profilering av produkter, enligt deras egen utsago. Informationen är ju inte nödvändigtvis referentgranskad, men den används ej heller i sådan utsträckning var det skulle behövas.


% vim: set tw=78:ts=2:sw=2:et:fdm=marker:wrap:wm=78:ft=tex
% vim: spell spelllang=sv
