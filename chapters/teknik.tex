\section{Teknik}

SmartElement bygger i stort sett på fem centrala tekniker, PHP som användts för att programmera progamlogiken, MySQL som databaslager var data lagras, Memcache som caching lager vilket tillåter att information som är tung att beräkna kan sparas mellan förfrågningar, dokument databsen MondgoDB för lagring av information om besökare och till sist JavaScript som används för att skapa det informationspaket om användaren som skickas till servern för användning vid filtreringen.

\subsection{PHP}

PHP är ett skriptspråk som utvecklats med målet att snabbt kunna skapa dynamiska webbsidor. \citep{phpmanual} Språket valdes för att det är enkelt att arbeta med och det var språket som företagets andra produkter va skrivna med. Det finns också en mängd information och färdiga kodpaket till förfogande för programmerare, vilket underlättar och försnabbar utvecklingen.

\subsection{MySQL}

För lagring av data valdes databasmjukvaran MySQL. En relationell databas passar sig bra för lagring av den information som systemet behandlar eftersom det handlar som entiteter som är starkt länkade till varandra.

\subsection{MongoDB}

Dokumentdatabsen MongoDB används för lagring av användarinformation. MongoDB har använder inte ett strikt schema för datalagring utan dokument i samma samling kan ha skiljande attribut associerade med sig, detta underlättar vidareutveckling av systemet genom att tillåtta legacy data att existera i databasen tillsammans med ny tills den gamla uppdateras genom vanlig operation.

\subsection{Memcache}

Eftersom SmartElements datamodeller är relativt tunga att bygga så bestämmdes det att ett caching lager skulle användas för att hålla dessa datamodeller och på så vis låta mjukvaran gå direkt till filtrerings logiken då en förfrågan behandlas.

Målet med cache lagret är att hålla sidorna i minnet på servern, och endast läsa från databasen när ett objekt byggs om i samband med en uppdatering. För detta ändamål valdes Memcache, en väletablerad mjukvara för caching. Memcache valdes för att det var en teknik som utvecklarna var vana vid samt för dess enkla modell av horisontell skalning, vilket skulle tillåta systemet att snabbt reagera på en växande kundskara.

\subsection{JavaScript}

För att kunna identifiera så mycket information som möjligt om användaren är det nödvändigt att köra kod på klienten innen innehållet kan presenteras. I dagens läge finns det endast ett skriptspråk som kan köras på godtycklig klient, och det är javascript, vilket valdes även för detta projekt. 

% vim: set tw=78:ts=2:sw=2:et:fdm=marker:wrap:wm=78:ft=tex
% vim: spell spelllang=sv
