\section{Processer}

För att kunna leverera anpassat webbinehåll till besökaren av en webbsida, måste systemet ha ett sätt att kunna identifiera information om denne som sedan kan användas för att filtrera bland det tillgängliga innehållet.

\subsection{Information om användaren}

SmartElement använder sig till stor del av statistik som systemet samlar ihop med hjälp av en JavaScript tag som körs i besökarens webbläsare då denne besöker en sida som använder systemet. Denna information skickas till back-end systemet som vidare processerar datan och sparar den i en dokument databas under ett beökarobjekt. På detta viset kan systemet komma ihåg användare mellan sessioner och på så vis skapa historik om användarens beteende och vanor, vilket i sin tur är till nytta för att hitta det bästa innehållet att presentera.

\subsubsection{Hänvisningsinformation}

Den första informationen som registreras om användaren är varifrån denne anländer till sidan. Webbläsare skickar oftast en så kallad referrer variabel när man navigerar till en webbsida genom att klicka på en länk, denna variabel innehåller addressen på sidan som visade länken.

Genom att spara denna information får man dels en indikation av varifrån besökare hittar till sidan, men även en indikation av vilken typs sidor den aktuella besökaren ofta besöker.

En nackdel är att hänvisningsinformation inte alltid finns tillgänglig, och avsaknad av denna information ger en inte ett definitivt svar om hur användaren hittat till sidan. Detta på grund av att hänvisningsinformation inte skickas t.ex. när man navigerar till webbsidan från en sida med krypterad anslutning, om användarens webbläsare har konfigurerats att inte sända informationen men även om användaren helt enkelt skrivit in webbsidans address i sin webbläsare och är en så kallad direkt träff. \citep{httprfc}

\subsubsection{IP Geolokalisering}

Nästa bit av information som samlas in om bseökaren är uppskattng av dennes position på basis av den information som registrerats för IP-addressen som förfrågan kommer från.

SmartElement använder sig av en databas som köps in av ett företag som specialiserar sig i att uppehålla så noggrann information som möjligt om var i världen IP-addresser egentligen är registrerade. Från denna databas söker systemet sedan fram information om besökarens läge. Information som är tillgänglig är bland annat vilket land användaren befinner sig i, vilken region inom landet samt vilken stad.

Trots att informationen i databasen ar av rätt hög kvalitet, kan man inte helt och hållet lite på informationene som genereras genom IP geolokalisering. Dels så är IP-addressen som sänds till servern inte nödvändigtvis besökarens egna IP-address då denne kan vara uppkopplad via en VPN-anslutning eller eventuellt använda sig av en proxy-server som inte vidare förmedlar ursprungsaddressen, detta betyder i sin tur att man får platsen var VPN- eller proxy-serverns IP är registrerad. Ett annat problem är att den information som finns tillgänglig beror på vad besökarens internet leverantör rapporterar, så även om IP-addressen är användarens egna så kan leverantören ha registrerat addressen på en annan plats än den var användaren befinner sig. 

\subsubsection{Besökarstatistik}

För varje sidvisning på en webbsida med SmartElement sparas det statistik. Användarobjektet som sparas i systemet innehåller information om hur ofta besökaren varit på webbplatesen, hur många sidor denne besökt inom webbplatsen samt hur ofta denne sett en specifik sida.

Genom att skapa statistik som är kopplad till användaren kan man generera information om användarens beteemde då denne använder webbsidan. Man kan räkna hur ofta denne besöker specifika delar av sidan, om den inte sett en viss sida på länge, om den börjat besöka en specifik del mer eller mindre o.s.v. All denna information kan hjälpa att söka fram information som kan tänkas relevant för besökaren.

De största problemen med statistik är att den är beroende av möjligheten att identifiera besökaren då denne återvänder till webbsidan. Som systemet är byggt för tillfället hänger detta på användningen av kakor som innehåller en unik id-nummer, vilket betyder att systemet tappar användaren så fort denne raderar denna kaka. Det finns sätt att rundgå detta genom 


\subsubsection{Tid}

Vid varje sidvisning registreras tiden för besöket på basis av den tid som användarens webbläsare rapporterar. Utöver tiden för sidvisningen så beräknar javascript tagen hur lång tid användaren spenderat på sidan under det pågående besöket.

Valet att registrera tiden som rapporteras av webbläsaren gjordes för att undvika problem med olika tidszoner och för att det är troligare att man vill göra ett beslut utgående från användarens tid, och eftersom tagen registrerar användarens tid så får systemet samtidigt möjligheten att registrera besökets längd.

\subsubsection{Användardefinierad information}

För att tillåta så flexibel användning som möjligt, tillåter SmartElement även att användaren själv definierar information som skickas till servern för processering. Detta sker genom att användaren lägger till egna dataelement i tagen då den inkluderas på sidan. Det finns två element som kan användas för detta ändamål, dels en samling med nyckel-värde par som kan användas vid processering av den pågående sidvisningen och dels ett element för insamling av användardefinierad statistik.

Genom denna teknik kan användaren i princip använda vad helst information som är relevant i dennes fall. Exempel på data som kunde vara intressant att spara är kundvagns inneåll, nån form av id för registrerade användare, aktiva kampanjer och annan information om webbsidasns status vid sidvisningen. Systemet sätter ingen begränsning på vad som skickas förutom att det måste sändas i formen av en samling av värden med nycklar.

\subsection{Filtrering}

För att välja ut vilket innehålls element som skall skickas till besökaren använder sig systemet av informationen som samlats in och ett eller flere filter som användaren definierat för de olika elementen som registrerats i systemet. Filtersystemet bygger på användningen av enkla test som i kombination med olika typers data bildar en fråga om en användare.

\subsubsection{Villkor}

I grunden bygger SmartElements filter system på 12 enkla villkorsfunktioner som utför en jämförelse mellan den information tagen skickat och den information som användaren sparat i filtret.

De tolv villkoren som kan användas samt de datatyper de kan hantera listas i tabell \ref{table:villkor}.

\begin{table}
    \begin{tabular}{|l|p{8cm}|l|}
    \hline
    Namn & Funktion & Datatyp \\
    \hline
    Större än & Testar om värdet är större än det i filtret & Alfanumerisk \\
    \hline
    Mindre än & Negation av större än villkoret & Alfanumerisk \\
    \hline
    Lika med & Testar om värdet är lika med det i filtret & Godtycklig \\
    \hline
    Inte lika med & Negation av lika med villkoret & Godtycklig \\
    \hline
    I samling & Testar om samlingen som registrerats i filtret innehåller värdet som sändts & Samling \\
    \hline
    Icke i samling & Negation av "i samling" villkoret & Samling \\
    \hline
    Innehåller & Testar om värdet i filtret innehåller värdet som sändts & Text \\
    \hline
    Innehåller ej & Negation av innehåller villkoret & Text \\
    \hline
    Börjar med & Testar om värdet som skickats börjar med värdet i filtret & Text \\
    \hline
    Slutar med & Testar om värdet som skickats slutar med värdet i filtret & Text \\
    \hline
    Tom & Testar om värdet som skickats är tomt & Text, Samling \\
    \hline
    Icke tom & Negation av tomhetsfiltret & Text, Samling \\
    \hline
    \end{tabular}
    \caption{Villkorstyper}
    \label{table:villkor}
\end{table}

Genom att kombinera dessa simpla villkor med den data som samlas in vid sidvisningar skapas filter som kan göra meningsfulla beslut om besökaren.

% vim: set tw=78:ts=2:sw=2:et:fdm=marker:wrap:wm=78:ft=tex
% vim: spell spelllang=sv
