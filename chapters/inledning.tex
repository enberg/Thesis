\section{Inledning}

Detta arbete dokumenterar systemet SmartElement som utvecklades för företaget Developer’s Helsinki Ab. Systemets syfte är att möjliggöra enkel konfiguration av anpassat innehåll för webbsidor, vilket tillåter webbsidors ägare att själva kunna konfigurera regler som bestämmer vad för innehåll som skall lyftas fram för olika användare.

\subsection{Målsättning och syfte}

Målsättningen med arbetet är att kartlägga den arkitektur som bäst stöder målet av ett effektivt system för innehålls-filtrering och leverans. Systemet är även lätt att vidareutveckla och utvidga i form av nya filter och nya algoritmer för filtrering.

Arbetet redogör för den information som systemet kan registrera om besökare till en webbsida. Olika problem relaterade till informationens pålitligheten beaktas. Utgående från den information som samlas in definieras 14 filtertyper som implementerats i systemet. Dessutom beskrivs hur filtren fungerar. 

Arbetet redogör även för den tekniska arkitekturen bakom systemet och hur den utformats. Mjukvaruarkitekturen förklaras och systemets uppbyggnad presenteras utgående från den centrala funktionaliteten, vilket är filtrering av innehåll.

\subsection{Avgränsningar}

Detta arbete är en redogörelse för SmartElements tekniska implementation, inte användningen av anpassat innehåll på webbplatser. Arbetet tar därmed inte ställning till hur anpassat innehåll skall användas, endast hur det kan implementeras.

Efter som det inte finns någon klar standard för vad som kan anses vara bra prestanda för ett liknande system, kommer arbetet inte att innehålla prestandaanalyser av systemet.

% vim: set tw=78:ts=2:sw=2:et:fdm=marker:wrap:wm=78:ft=tex
% vim: spell spelllang=sv
