\begin{abstract}{swedish}
  Avsikten med detta arbeet är att redogöra för implementationen av ett system
  för automatiserad anpassning av webbinnehåll.

  Arbetet presenterar i korthet anpassning av webbinnehåll och presenterar sedan
  systemet SmartElement som utvecklats för att tillåta webbsideupprätthållare att
  enkelt kunna implementera anpassat innehåll på sin sida.

  Olika sätt att göra bedömningar om besökare på en webbsida studeras. Genom
  att redogöra vilken sorts information systemet kan samla in om en besökare
  definieras filtertyper som kan användas för att filtrera innehåll på basis av
  informationen.

  Utöver detta redogörs det för den tekniska arkitektur som valts ut för systemet,
  samt hur de val som gjorts vid planeringen söder systemets funktionalitet.

  Slutligen diskuteras några områden var systemet skulle kunna vidareutvecklas.
  \\
  \\
\end{abstract}

\begin{abstract}{english}
  The purpose of this thesis is to describe the implementation of a system for
  automatic personalization of content on a website.

  The thesis gives a brief introduction to the concept of website personalization
  and presents the SmartElement system, a tool that has been developed to allow
  website administrators to easily utilize personalization of their content.

  The thesis takes a look at what kind of deductions can be made about a visitor.
  By defining a list of information that the system can collect about visitors,
  a list of filters is defined, which can be used for filtering the content on
  a website.

  The technical architecture behind the system is also presented. Some of the choises
  made during the design of the architecture, and the impact of theese choises,
  are also presented.

  Finally a list of proposed areas of furher development is made, documenting areas
  where the system could be improved.
  \\
  \\
\end{abstract}

% vim: set tw=78:ts=2:sw=2:et:fdm=marker:wrap:wm=78:ft=tex
% vim: spell spelllang=sv
